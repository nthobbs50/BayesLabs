\batchmode
\makeatletter
\def\input@path{{\string"/Users/Tom/Documents/NSF Statistics Workshop/BayesLabs/\string"}}
\makeatother
\documentclass[11pt,american]{article}
\usepackage[T1]{fontenc}
\usepackage[latin9]{inputenc}
\usepackage[letterpaper]{geometry}
\geometry{verbose,tmargin=1in,bmargin=1in,lmargin=1in,rmargin=1in,footskip=1cm}
\usepackage{array}
\usepackage{float}
\usepackage{amsmath}
\usepackage{amssymb}
\usepackage{setspace}
\usepackage[authoryear]{natbib}
\doublespacing

\makeatletter

%%%%%%%%%%%%%%%%%%%%%%%%%%%%%% LyX specific LaTeX commands.
%% Because html converters don't know tabularnewline
\providecommand{\tabularnewline}{\\}

%%%%%%%%%%%%%%%%%%%%%%%%%%%%%% Textclass specific LaTeX commands.

%%%%%%%%%%%%%%%%%%%%%%%%%%%%%% User specified LaTeX commands.
\makeatother
\usepackage{graphicx}
\usepackage{lineno,setspace}
%\linenumbers
\usepackage[small,compact]{titlesec}
\usepackage[small,it]{caption}
%\addtolength{\textfloatsep}{-20mm}
%\doublespace
\addtolength{\belowcaptionskip}{-3mm}
\addtolength{\abovecaptionskip}{-3mm}
\usepackage{enumitem} % load the package
\usepackage{bm}
\usepackage{calc}
\usepackage{rotating}

\DeclareMathOperator{\dbin}{binomial}
\DeclareMathOperator{\dpois}{Poisson}
\DeclareMathOperator{\dnorm}{normal}
\DeclareMathOperator{\dlnorm}{lognormal}
\DeclareMathOperator{\dgamma}{gamma}
\DeclareMathOperator{\dunif}{uniform}
\DeclareMathOperator{\dmultinom}{multinomial}
\DeclareMathOperator{\dbeta}{beta}
\DeclareMathOperator{\ddirch}{Dirichlet}
\DeclareMathOperator{\dbern}{Bernoulli}



\addtolength{\intextsep}{-3mm}


\usepackage{url}
%\usepackage[tablesfirst,notablist]{endfloat}
\usepackage{fancyheadings}
\pagestyle{fancy}
\input{"/Labs/middle_header.txt"}
\chead {BayesLabs}

\makeatother

\usepackage{babel}
\usepackage{Sweave}
\begin{document}

\section*{Instructions for Using BayesLabs}

This repository contains a series of laboratory exercises designed
to support an introductory course in Bayesian modeling for graduate
students, post-docs, faculty, and research scientists. The material
can be adapted for use in a 10-day intensive workshop format or a
more traditional semester-long course. Examples are drawn from social
science and ecology. The materials were developed by N. Thompson Hobbs,
Mevin B. Hooten, Christian Che-Castaldo, Mary Collins, Kiona Ogle,
and Maria Uriarte with support from the National Science Foundation
(awards DBI 1052875 and DEB1145200). The labs were designed to compliment
reading and lectures based on Hobbs, N. T., and M. B. Hooten. 2015.
Bayesian models: a statistical primer for ecologists. Princeton University
Press, Princeton, N.J. (Table 1).

Each folder in the repository contains an R markdown file with switches
that toggle output of .html files for exercises alone and exercises
with answers. Using these switches is explained in the R markdown
files. The files \texttt{/Labs/title.txt }and \texttt{/Labs/subtitle.txt
}specify the course name on all materials, allowing users to change
the title to match the name of the course they are teaching. We ask
that the logo acknowledging NSF support remain unchanged by users. 

The \texttt{/Admin }folder contains instructions for pulling the repository
and for creating an R package containing the data library required
for the exercises.

\begin{table}[H]
\caption{Laboratory exercises supporting an introductory course in Bayesian
modeling. Readings are chapters from Hobbs and Hooten 2015. Exercises
are arranged with introductory topics at the top of the table and
more advanced topics toward the bottom.}

\begin{tabular}{|>{\raggedright}p{1.9in}|>{\raggedright}p{3.5in}|>{\centering}p{1in}|}
\hline 
Folder in \texttt{/Labs} &
Topics and challenges &
Reading\tabularnewline
\hline 
\hline 
\texttt{Probability} &
Rules of probability. Factoring joint distributions. Probability distributions.
Marginal distributions. Moment matching. &
Chapter 3\tabularnewline
\hline 
\texttt{HemlockLightExample} &
Likelihood functions. Computing total likelihoods from multiple observations.
Using prior information in the likelihood framework. &
Chapter 4\tabularnewline
\hline 
\texttt{ConjugatePriors} &
Find parameters of posterior distributions using conjugacy between
likelihoods and priors.  &
Chapter 5.3\tabularnewline
\hline 
\texttt{BayesTheorem} &
Compute the likelihood, prior, and marginal distribution of the data
and assemble them to compute the posterior distribution. &
Chapter 5.1, 5.2\tabularnewline
\hline 
\texttt{MCMC1} &
Introduction to Markov chain Monte Carlo using Gibbs sampling for
normal mean and variance. &
Chapter 7\tabularnewline
\hline 
\texttt{MCMC2} &
Accept-reject sampling using Metropolis-Hastings algorithm. &
Chapter 7\tabularnewline
\hline 
\texttt{JAGSPrimer} &
A tutorial on JAGS and rjags for implementing Markov chain Monte Carlo. &
none\tabularnewline
\hline 
\texttt{JAGSProblem} &
Using JAGS and rjags. &
none\tabularnewline
\hline 
\texttt{ModelBuilding} &
Word problems to teach drawing directed acyclic graphs and using them
to write posterior and joint distributions. &
Chapters 6, 10, 11, 12\tabularnewline
\hline 
\texttt{MultiLevelModels} &
Model building and computation for group-level intercepts and slopes. &
Chapter 6.2.2\tabularnewline
\hline 
\texttt{PosteriorPredictiveChecks} &
Using data simulation to test for lack of fit. Computing Bayesian
p values. &
Chapter 8.1\tabularnewline
\hline 
\texttt{SwissBirds} &
Occupancy models. Zero inflation. Derived quantities. &
Chapter 6.2.3\tabularnewline
\hline 
\texttt{ModelSelection} &
Evaluating evidence for alternative models using the DIC, WAIC, and
posterior-predictive loss. &
Chapter 9\tabularnewline
\hline 
\texttt{DynamicModels} &
State space models. Predictive process distributions. Forecasting. &
Chapter 8.4, 8.5\tabularnewline
\hline 
\texttt{MetaAnalysis} &
Combining information from multiple studies in priors and likelihoods. &
Chapter 4\tabularnewline
\hline 
\end{tabular}
\end{table}

\end{document}
