\documentclass[12pt, oneside]{article}     

% packages
\usepackage[margin=1in]{geometry}                  	
\geometry{letterpaper}                   	
\usepackage{graphicx}				
\usepackage{amssymb}
\usepackage{amsmath}                                                                                                                                                                                                                                                                                                                                                                                                                                                                                                                                                                                          
\usepackage{bm}
\usepackage{float}
\usepackage[nodisplayskipstretch]{setspace}
\usepackage{enumitem}
\usepackage{fancyheadings}
\usepackage{mathtools}
\usepackage{csquotes}
\usepackage[colorlinks = true,
            linkcolor = blue,
            urlcolor  = blue,
            citecolor = blue,
            anchorcolor = blue]{hyperref}

\newcommand{\link}[3][blue]{\href{#2}{\color{#1}{#3}}}%

% Configuration
\setstretch{1.5}
\pagestyle{fancy}
\input{"../middle_header.txt"}
\chead{538 Presidential Election Model}
\title{\vspace{-1cm}Capstone Exercise}

% Custom commands
\newif\ifanswers
 \answerstrue % comment out to hide answers

\begin{document}
\maketitle

\thispagestyle{fancy}

\section*{Motivation}
Nate Silver, who used Bayesian models to accurately predict Barack Obama's 2008 election is at it again in 2016 (\link{http://fivethirtyeight.com/features/a-users-guide-to-fivethirtyeights-2016-general-election-forecast/}{official methods link})!  His model results in the below estimates, which, by this point, should look pretty familiar.  
\vspace{5mm}
\begin{center}
\includegraphics[scale=.5]{HillzPosteriorsYo.png}
\end{center}

\newpage

\section*{Problem}

\begin{enumerate}
\item Write out the DAG and joint distribution for this problem.

\ifanswers
\begin{center}
\includegraphics[scale=.4]{DAG}
\end{center}
\noindent where $y$ is the number of votes for Hillary Clinton, in the $j^{th}$ poll, in week $t$, in the $i^{th}$ state.
\begin{eqnarray*}
\big[\boldsymbol{\alpha},\boldsymbol{\beta}, \mathbf{p} \mid \mathbf{y}\big] \propto & \prod_{i=1}^{51}\prod_{j=1}^{N_{j}}\prod_{t=1}^{T} \big[y_{ijt} \mid p_{ijt}\big]\big[p_{ijt} \mid \alpha_{it},\beta_{it}\big]
\big[\alpha_{it}\big]\big[\beta_{it}\big]
\end{eqnarray*}
\fi 

\item How would you decide who wins each state? Who wins the national vote? Remember candidates must accumulate 270 of 538 state-level electoral votes to win the election. To get started with this question, consider taking draws from a derived quantity from your model, remembering that we are interested in expressing state and national results as random variables.  Essentially, we want you to recreate the posterior distributions shown in the earlier figure. 

\ifanswers
\begin{center}

$\phi_{it}=\cfrac{\alpha_{it}}{\alpha_{it}+\beta_{it}}$
\vspace{5mm}

$V_{it} = \textrm{binomial}\big(N_{it}, \phi_{it}\big)$

where $N_{it}$ is the number of likely voters in the $i^{th}$ state and the $t^{th}$ week.

if $V_{it} > \cfrac{N_{it}}{2}$ then $Z_{it}=1$ else $Z_{it}=0$

$E_{t} = Z_{it} \times$ the number of electoral votes $= \Sigma_{i=1}^{51} E_{it}$

$H=1$ if $E_{it} \geq 270$, then Clinton wins.

$H=0$ if $E_{it} < 270$, then Trump wins.
\end{center}
\fi

\item As the election season progresses, there is more and more information to use to inform priors.  Think about how might you incorporate a weight, $w$, for each of the priors.  We will discuss this as a group.
\vspace{-5mm}
\ifanswers
\begin{eqnarray*}
\alpha_{it} &= \Sigma^J_{j=1} w_{ij} a_{ijt-1}\\
\beta_{it} &= \Sigma^J_{j=1} w_{ij} b_{ijt-1}
\end{eqnarray*}
\fi
\end{enumerate}
\end{document}

